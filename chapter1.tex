\chapter{Chapter 1 - Introduction}
This chapter will introduce the reader to this report and will give insights regarding motivation, scope and contribution. In addition to this, the objective and outline will be presented, and this chapter along with Chapter 2 - Background will serve as a fundament for this report as a whole.
\section {Motivation}
%freemium
\newline
\\
%verdiøkning av kart
\section {Scope \& objective}
To achieve the goal of proposing a business model based on the freemium it is necessary to narrow down the aspects considered and have a clear view of the objectives of this report:
\newline
\\
\textbf{Scope: }MazeMap already has a established base of customers
\newline
\\
\textbf{Objectives: }This report aims to investigate if the freemium model is feasible and if it is capable of accelerate customer acquisition. In doing so a business model will be proposed. The objectives can be summarized in three points:
\begin{enumerate}
\item Investigate if there is a demand for such a service in the B2B\&C market if it is free in its basic form
\item Will it accelerate customer acquisition?
\item Suggest a business model based on the premise of "free" and of the findings
\end{enumerate}
\section {Contribution}
The research in this report is mainly aimed at business owners who wish to expand their offerings to their respective customers. through a mapping service such as MazeMap. The key contributions will be the market survey engineered to target the problem at hand, and the proposed business model aimed at aiding MazeMap in determining whether a free model could accelerate customer acquisition which in turn is the main bottleneck in terms of sales.
\section {Related work}
MazeMap (along with Wireless Trondheim) have provided several projects and master theses in cooperation with NTNU. In relation to this report, three master theses protrudes as being particularly interesting: "International Business Potential for Analytics of Room Utilization" by Binde, Karl 2015, "Business Potential for Analytics of Data from Wi-Fi Networks" by Bergendal, Petter 2014 and "Campusguiden" by Halvorsen, Christian 2011 \cite{petterbergendal2014}\cite{karlbernhoffbinde2015}\cite{christianhalvorsen2011}. The first of these focuses both on the technical and financial aspects, with emphasis on the financial. Through surveys sent out to a plethora of HEIs (Higher Education Insitutions), it aimed to propose a business model based on the findings from the survey, and as such its scope differs from this report in that it is a narrower scope and its methodology is more quantitative than what is found here. The second focuses on MazeMap as well, but the target was exclusively shopping malls with surveys handed out to respective shopping malls. The focal point of the report was finding a way to generate value from analytics of Wi-Fi data, and features most discussed were flow and counting of customers and duration of visits. In contrast to this report, the focus here is mainly Norwegian arenas where MazeMap can generate value for customers of various types of businesses. The third focuses mostly on an early version of MazeMap from a more technical point of view, but it also encompasses financial aspects. However, its methodology is solely qualitative and it focuses mostly on the indoor navigation system, a different topic than what is discussed in this report. 
\newline
\\
A master thesis "Freemium for large enterprises" by Jepson, Lundin 2009 \cite{annajepsonelinlundin2009} also encompasses the financial aspect of freemium, which is established in the B2C market, but not so much in the B2B market. This thesis looks at the Swedish company Teleopti and how it may generate leads in the large enterprise market, and how a free product should stimulate demand for a premium version, as well as the importance of being first-to-market. As such, it encompasses several aspects of this report, and as such it will be scrutinised further serving as a case study in this report.
\section {Motivation}