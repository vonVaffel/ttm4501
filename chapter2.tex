\chapter{Chapter 2 - Background}
In this section we will review a selection of terms used throughout this paper, in order to gain a fundamental understanding of how the 
\section{Freemium: The 0-cost model}
\subsection {Etymology}
The word "Freemium" is a portmanteau of the words "Free" and "Premium". The term was first coined by Jarid Lukin in 2006 in response to Fred Wilson's proposal to name his favorite business model which he described as follows~\cite{freemiumdef}:
\begin{displayquote}
\textquote{Give your service away for free, possibly ad supported but maybe not,
acquire a lot of customers very efficiently through word of mouth,
referral networks, organic search marketing, etc, then offer premium
priced value added services or an enhanced version of your service to
your customer base.} Fred Wilson, 2006 \cite{barbarafindlayschenck2007}
\end{displayquote}
\subsection{What is the freemium business model?}
Up until the mid-nineties, giving away a product for free was more or less unheard of. While razor-blade powerhouse Gillette sold cheap razors and made money primarily on the blades, it wasn't until the advent of the Internet that made low-cost online sales distribution come into fruition. One of the first actors to realise this was Adobe who released their Adobe Acrobat PDF-reader for free in 1994. Another company, Macromedia, released their Shockwave Player for free the following year. Both of these became industry standards in their respective segments, making money by selling more feature rich versions of their products\cite{katherineheires2007}.
\newline
\\
The focal point of this business models is that a lesser portion of the customer base subsidise the majority. What enables this is a digital market where marginal distribution- and production costs are close to nothing, and the idea is to charge a small part of the user base as opposed to charging the entire user base for a smaller sum \cite{chrisanderson2008}. Former VP of Global Sales at Skype, Jonas Kjellberg, talks about frequency being an integral part in Skype's success: You try to reach a larger customer base by "bombarding" customers with a product as opposed to a more restrained approach in customer acquisition where a potential customer would get more attention. In terms of a sales pipeline\footnote{An approach to sales using scientific and mathematical principles to achieve certain goals during the sales process \cite{selden1996}} it means that a higher frequency of potential customers in, will ultimately lead to a higher number of paying customers overall.
\newline
\\
There are however different definitions of what freemium constitutes.: Author and journalist Chris Anderson, when asked by a CEO of a SaaS company, defined four different freemium models \cite{chrisanderson2012}:
\begin{itemize}
\item \textbf{Time Limited: }The product is free for 30 days, then the user has to pay for it. Upsides of this model is that it is easy to implement with a low risk of cannibalization\footnote{ “The process by which a new product
gains sales by diverting them from an existing product”\cite{heskettj1976}}. However, due to customers gaining no benefit after 30 days, they might not be willing to commit enough to give the product a chance.
%legg inn tanker rundt dette
\item \textbf{Feature limited: }The product exist as a free and a paid version, where the paid version is more feature-rich. The benefit from this model lies in maximizing reach, and customers are not being misled when they pay for the full version. These customers also tend to be more loyal and less price sensitive. There are however some inherent downsides in that one needs to create two versions of essentially the same products. Making the free version too feature rich will lead to less people opting for the full version. Conversely if too few features are put in the free version, users will have no incentive to switch to the full version.
\item \textbf{Seat limited: }In this model, the product can be used by a certain number of people for free, but any users above this limit has to pay. Advantages include ease of implementation and understanding, while the downside is that this model might cannibalize the low end of the market.
\item \textbf{Customer type limited: }The product is free for smaller \& younger enterprises, while bigger enterprises and companies have to pay. This assures that a company's ability to pay is enforced, and enables growth in these smaller companies. It is however complicated to ascertain what constitutes a small business, and when a company is big enough to have to pay for the product. 
\end{itemize}
The latter being a particularly interesting case, as this type is used in the B2B market, where Microsoft's BizSpark service offers free software to companies that has existed for less than 5 years with less than \$1 million in revenues, \cite{microsoft2015}.

\section{B2B: Business to Business}
\section{B2C: Business to Consumer}
\section{Implications of B2B vs. B2C}
\section{B2B\&C}

