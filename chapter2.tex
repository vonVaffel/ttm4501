\chapter{Chapter 2 - Background}
In this section we will review a selection of terms used throughout this paper, in order to gain a fundamental understanding of how the 
\section{Freemium: The 0-cost model}
\subsection {Etymology}
The word "Freemium" is a portmanteau of the words "Free" and "Premium". The term was first coined by Jarid Lukin in 2006 in response to Fred Wilson's proposal to name his favorite business model which he described as follows~\cite{freemiumdef}:
\begin{displayquote}
\textquote{Give your service away for free, possibly ad supported but maybe not,
acquire a lot of customers very efficiently through word of mouth,
referral networks, organic search marketing, etc, then offer premium
priced value added services or an enhanced version of your service to
your customer base.} Fred Wilson, 2006 \cite{barbarafindlayschenck2007}
\end{displayquote}
\subsection{What is the freemium business model?}
Up until the mid-nineties, giving away a product for free was more or less unheard of. While razor-blade powerhouse Gillette sold cheap razors and made money primarily on the blades, it wasn't until the advent of the Internet that made low-cost online sales distribution come into fruition. One of the first actors to realise this was Adobe who released their Adobe Acrobat PDF-reader for free in 1994. Another company, Macromedia, released their Shockwave Player for free the following year. Both of these became industry standards in their respective segments, making money by selling more feature rich versions of their products\cite{katherineheires2007}.
\newline
\\
The focal point of this business models is that a lesser portion of the customer base subsidise the majority. What enables this is a digital market where marginal distribution- and production costs are close to nothing, and the idea is to charge a small part of the user base as opposed to charging the entire user base for a smaller sum \cite{chrisanderson2008}. Former VP of Global Sales at Skype, Jonas Kjellberg, talks about frequency being an integral part in Skype's success: You try to reach a larger customer base by "bombarding" customers with a product as opposed to a more restrained approach in customer acquisition where a potential customer would get more attention. In terms of a sales pipeline\footnote{An approach to sales using scientific and mathematical principles to achieve certain goals during the sales process \cite{selden1996}} it means that a higher frequency of potential customers in, will ultimately lead to a higher number of paying customers overall.
\newline
\\
There are however different definitions of what freemium constitutes.: Author and journalist Chris Anderson, when asked by a CEO of a SaaS company, defined four different freemium models \cite{chrisanderson2012}:
\begin{itemize}
\item \textbf{Time Limited: }The product is free for 30 days, then the user has to pay for it. Upsides of this model is that it is easy to implement with a low risk of cannibalization\footnote{ “The process by which a new product
gains sales by diverting them from an existing product”\cite{heskettj1976}}. However, due to customers gaining no benefit after 30 days, they might not be willing to commit enough to give the product a chance.
%legg inn tanker rundt dette
\item \textbf{Feature limited: }The product exist as a free and a paid version, where the paid version is more feature-rich. The benefit from this model lies in maximizing reach, and customers are not being misled when they pay for the full version. These customers also tend to be more loyal and less price sensitive. There are however some inherent downsides in that one needs to create two versions of essentially the same products. Making the free version too feature rich will lead to less people opting for the full version. Conversely if too few features are put in the free version, users will have no incentive to switch to the full version.
\item \textbf{Seat limited: }In this model, the product can be used by a certain number of people for free, but any users above this limit has to pay. Advantages include ease of implementation and understanding, while the downside is that this model might cannibalize the low end of the market.
\item \textbf{Customer type limited: }The product is free for smaller \& younger enterprises, while bigger enterprises and companies have to pay. This assures that a company's ability to pay is enforced, and enables growth in these smaller companies. It is however complicated to ascertain what constitutes a small business, and when a company is big enough to have to pay for the product. 
\end{itemize}
The latter being a particularly interesting case, as this type is used in the B2B market, where Microsoft's BizSpark service offers free software to companies that has existed for less than 5 years with less than \$1 million in revenues, \cite{microsoft2015}.

\section{Business Markets: B2C \& B2B}
B2B, or business-to-business, is the case when a business transaction occurs between two or more companies \cite{jewels2001towards}. Compared to the B2C counterpart, B2B transactions often involve more than one person, often called the decision making unit \cite{paulhaguenickhaguematthewharrison}. These units may be constantly changing and each individual will possibly have their own interests and motivations. Whereas consumers rationality can be questioned in B2C markets, it is usually assumed that buyers in the B2B market are rational actors, meaning they will be more concerned with what is needed as opposed to what is wanted. Regarding the types of products, the B2B paradigm often see more complex products that requires certain skills to operate. Furthermore, industrial products often require interoperability with existing products, and thus have to been seamlessly integrated and modified to fit the current systems in place. For instance, a  company might not be concerned with the looks of a laptop, but will be more concerned with longevity, security, performance etc. Consumers on the other hand might also care greatly for the aforementioned qualities, but they might also value the design of a laptop equally to performance. 
\newline
\\
When it comes to units sold, this number tend to vastly differ in B2B from B2C. Consumers are often times limited by how many units they can single-handedly consume, but this does not apply to businesses. The Pareto rule, or the 80:20 rule applies to businesses in most cases, where 80\% of the revenue comes from 20\% of the customers. It is feasible to map how much a consumer will consume of a certain product during a period of time, but this may not be feasible in the business perspective: The difference in spending between the largest and smallest buyer in a business context will likely be much larger than that of consumers. It can be summarized that in B2B markets there are few customers with largely varying purchasing volumes, whereas in B2C markets the purchasing volumes are largely the same, but the amount of customers varies by a large amount.
\newline
\\
Due to a smaller customer base and less varying behavior of clients in B2B markets, market segmentation can be described by four different market segments:
\begin{itemize}
\item \textbf{Price-oriented: }Looks for a no-frills business experience without a need for potentially superfluous services and values the price of a product more than other factors. Typically, smaller businesses fall into this category.
\item \textbf{Quality-oriented: }Seeks the best possible product and has high willingness to pay. It can be characterised by high margins and is often a medium/large-sized company. 
\item \textbf{Service-oriented: }Has high requirements for reliability and quality. Additionally, aftersales and delivery can be seen as important factors as well. Usually, time-sensitive markets with businesses of any size with a high sales volume will fall under this category. 
\item \textbf{Partnership-oriented: }Often concerns key accounts. Values trust and reliability and views the supplier of products as a strategic partner. 
\end{itemize}
Arguments can be made that customer relationships are more intimate in a B2B setting. It is easier to maintain a relationship with a handful of representatives (as opposed to every consumer that has procured a product) with a more direct channel between partners. While a consumer might have need for aftersales support , it is on a more diminished scale than in B2B markets. A photocopier in an office for example will probably be used more than a photocopier in the hands of a consumer, as it will probably be in need of being serviced regularly. As businesses in the B2B market tend to have a fewer number of customers, losing a customer can sometimes be disastrous, and conversely having a reliable customer or partner can be of great value to businesses. Since businesses can be seen as more rational actors than consumers, it follows that the need for innovation is less apparent in B2B markets, as trends do not need to be created but followed. This induces less risk for businesses in the B2B market, and more risk-averse decisions can be made. 
\section{B2B\&C}
Business-to-business-and-consumer differs from both B2B and B2C in that the market segment is more akin to the B2C market. The users of a service or a product is not the business itself, but rather the users or patrons of various establishments. For example, in the case of MazeMap, it is implemented at the university NTNU. NTNU has purchased MazeMap as a software-as-a-service product, but NTNU itself isn't the main benefactor and user of the product; the visitors, students and personnel are. That is not to say establishments themselves do not generate value from such a proposition: Hospitals for instance, can benefit greatly of an indoor mapping solution, by providing visitors and personnel with crucial information as to where they are supposed to meet for an appointment. If as little as 0,1\% more patients were able to be on time for their appointments, it would be cost beneficial for the hospital \cite{mazemaphospitals2015}. Additionally, in Great Britain missed appointments in the health sector has induced losses of £800 million per year, a cost that can be remedied by providing patients with an indoor mapping solution \cite{lucyjohnston2012}.

