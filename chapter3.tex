\chapter{Chapter 3 - Methodology}
This chapter will give the reader insights on the methodology of the research, what is under scrutiny and how these lead to the result of this report
\section{The Freemium Model in Indoor Maps Survey}
In order to obtain some empirical evidence in trying to ascertain how well a freemium model in indoor maps will be received by potential customers, a survey was conducted. The survey was done at a national level, where a leading figure and/or a property manager were asked to answer a survey estimated to take between five and ten minutes. 
The institutions surveyed were shopping malls, HEIs, hospitals and miscellaneous (airports, sporting arenas and museums).%The areas of interest were to find out the interest for an indoor map system, how a free system would be received and willingness to pay for additional value added services.
\newline
\\
Initially, the plan was to contact a CEO or a leading figure within each institution, but as the initial e-mail was sent out, a lot of respondents were oblivious as to what they were supposed to answer. Respondents were then informed (and subsequently the contact letter was changed) to look past the jargon used in describing the survey and try to answer the questions to their best effort. Some of the respondents were unable to answer the survey, as it was formed using Google Forms, which incidentally is blocked in hospitals in Norway. Respondents were then contacted by phone where the surveyor filled in the answers. The first contact letter was sent 6th. of October and the last at 20th. of October. 
\subsection{The Purpose of the survey}
The main goals of the survey were to see if there was an interest in an indoor mapping system, how a free system would be received and lastly to look at the interest for additional value added services. As described in Chapter 2 - Background, the latter being of great importance as it can be seen as an important revenue stream in the freemium model. Respondents were also asked if they had an existing indoor map system, and if a switch to a freemium model were desired over the existing modus operandi. Moreover, respondents were asked in case of not desiring an indoor mapping system what underlying factors that determined this.
\subsection{Reponse Rate, Potential Risks and Difficulties}
The main concerning when forming the survey and using it as the backbone of this report was the response rate. 
\newline
\\
As means to get as many responses as possible, the invitation letter was changed iteratively, and it was made clear that the respondents should not have to answer the survey, should they desire to not do so. They were also informed that a short interview could take place instead of answering the survey online, but only one respondent chose to take this course of action. The length of the survey was purposely made rather short, as it being answerable in five to ten minutes meant that otherwise busy leading figures might have the time to answer the survey.
\newline
\\
As previously mentioned finding the right person to contact proved difficult and the questions themselves required some authority on the issue of potentially acquiring an indoor mapping system. In total, 60 institutions were contacted, but the response rate was rather low at merely 14 responses.