\chapter{Chapter 3 - Background}
In this section we will review a selection of terms used throughout this paper, in order to gain a fundamental understanding of how the 
\section{Freemium: The 0-cost model}
\subsection {Etymology}
The word "Freemium" is a portmanteau of the words "Free" and "Premium". The term was first coined by Jarid Lukin in 2006 in response to Fred Wilson's proposal to name his favorite business model which he described as follows~\cite{freemiumdef}:
\begin{center}
\textit{"Give your service away for free, possibly ad supported but maybe not,
acquire a lot of customers very efficiently through word of mouth,
referral networks, organic search marketing, etc, then offer premium
priced value added services or an enhanced version of your service to
your customer base."}\\
Fred Wilson
\end{center}
\subsection{What is the freemium business model?}
Up until the mid-nineties, giving away a product for free was more or less unheard of. While razor-blade powerhouse Gillette sold cheap razors and made money primarily on the blades, it wasn't until the advent of the Internet that made low-cost online sales distribution come into fruition. One of the first actors to realise this was Adobe who released their Adobe Acrobat PDF-reader for free in 1994. Another company, Macromedia, released their Shockwave Player for free the following year  
\section{B2B: Business to Business}
\section{B2C: Business to Consumer}
\section{Implications of B2B vs. B2C}
\section{B2B\&C}