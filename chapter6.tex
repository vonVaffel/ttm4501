\chapter{Chapter 6 - Results and Discussion}
This chapter will look more closely at the answers provided by respondents from the survey in chapter 4. 
\section{Individual responses}
This section contains individual elaborations in relation to the questions asked in the survey. The reader is referred to chapter 4 for a thorough review of the survey in question.
\subsection{Headmaster of a Norwegian University}
Though short in length, this particular representative already had an existing indoor mapping system (in this case MazeMap) employed at their university. As the survey had mostly optional questions to defer the respondent from potentially not answering at all, some answers where shorter than others, this being a case of this. It was reported however, that the current solution was satisfactory, and that integrated systems in conjunction with an indoor mapping solution was of great use to the university. A point was also made that integration with a calendar was especially important.
\subsection{Headmaster of a Norwegian College}
This college is one of the smaller ones in Norway with around 4000 students and a staff of 300. At the time of the survey, this respondent stated that floor planning were made available on line as the main means of finding a particular indoor location. They didn't employ a indoor mapping system, but they were considering it. Furthermore, they were also positive towards purchasing additional services, and integration with a timetable was their highest ranking value-added service of the five, whilst integration with SMS, own apps and own systems scored low; 1 and 2 respectively.
\subsection{Engineering- and Planning Manager of a Norwegian Hospital - Interview}
This respondent being a hospital employee was unable to answer the survey online, due to Google being a restricted domain at Norwegian hospitals. An interview with the person in question was therefore conducted, where answers were given to the surveyor who entered them in the survey. Regarding Q1 they reported no indoor mapping system currently in place: Most directions were given manually to patients, through a discharge letter, with subsequent signage at the hospital. They were at the time however, considering an indoor mapping system, citing St. Olavs Hospital as an inspiration as the latter has an indoor mapping system in MazeMap \cite{st.olavshospital2013}. They were also the only respondent to give a figure regarding how much they were willing to spend on such a system, citing \textit{"if intelligent, some NOK 100.000"}. Regarding Q5, they were sceptical about a free system, stating that \textit{"nothing is "free" "}. The tone changed however, as the respondent stated that it would be advantageous to potentially try out a system beforehand, to test the foundations and mapping out potentially needed services. As was stated by other respondents, the total cost of a product is of concern, especially in the case of government procurement as hospitals are prone to. The respondent also raised an important point in that the person concerned was personally convinced that a freemium system would be easier to implement: This due to there being several key people that are part of the decision process in public as well as private enterprises, and it would possibly be more difficult to argue against procuring a system that is free in its basic form. 
\newline
\\
In asking Q6 \& Q7 regarding the willingness to pay for additional services, the respondent stated that all services were equally interesting in their own right, and granted each of the five services five out of five, emphasizing the importance of integration with SMS and timetables, enabling patients and personnel alike to have a better bearing on where they are supposed to be at a given time. When asked to comment freely on the proposition of a freemium-based indoor mapping system, the respondent replied that it would potentially reap several benefits for the hospital: They foreshadowed a reduction in the amount of manual work that has to be done in the case of patients, and stated that in the case of discharge letters, a QR-code with a link containing directions would be a good improvement over the current system. This in turn would spare reduce the need for waiting room area, as the person in question stated as a current issue.

\subsection{Chief of Construction and Technical Facilities at a Norwegian Airport}
This response came from the person mentioned in the headline of this subsection. The airport is one of the smaller ones compared to other Norwegian airports and is located in the northern part of Norway. In 2014 they reported that over 1,4 million passengers travelled through this airport \cite{avinor2014}. They reported having no customer-oriented mapping systems, only the necessities i.e. maps showing escape routes, fire alarms and fire extinguishers. They also reported an initial disinterest in an indoor mapping solution stating the size of the establishment as an important factor:
\begin{displayquote}
\textit{"We are a small airport in European context, thus finding your way is almost self-explanatory and all restaurants and shops have some form of signage. However, topical content would be more relevant than signs. Furthermore, an indoor mapping solution would probably be more interesting for the shop owners, as I can see them wanting potential customers to know where the stores are located."}
\end{displayquote}
Regarding Q5 on the premise of a free system the respondent replied that the layout would be of importance. This can be interpreted as this establishment was concerned with the actual look of the front-end of an indoor mapping system, but since no further answer was given this is speculative. In terms of services offered, they were considering an SMS service where the flier would get information about time to boarding, how far away one is from the correct terminal etc. Given their initial disinterest, the answer of Q7 was particularly interesting in this case, as they stated "demand" as an important underlying factor as to why an indoor mapping system was not of interest. The highest scoring parts of Q8 were the rankings of integration with SMS, timetable and own apps (4), with updating of maps and indoor navigation scoring the lowest with 1 and 2 respectively. As stated previously, low scores were to be expected on the question regarding updating of maps, as this might have been unclear for the respondents. 
\newline
\\
To argue their stance on not seeing the demand for an indoor mapping system, the respondent claimed that 5\% of the population accounts for 90\% of airline tickets, indicating that most frequent fliers are quite accustomed to their local airport. The respondent also pointed out that they experienced \textit{"that signs disappear in the jungle of signs at an airport. From our experience, intuitive screens with information is a better fit for an airport".} This may however, indicate a wish from the respondent to actually employ an indoor mapping system as the described functionality is readily implemented by an indoor mapping solution \cite{mazemap2015}.